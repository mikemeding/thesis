\chapter{Server Side Code Appendix}
\paragraph{}
Included in this appendix is the full source code written to run WebSlicer on a Wildfly application server.
The section names are the file locations as they exist from the top level directory.

% -----------------------------------------------------------------------------------------------------------------------------------------------------------
% CONFIG CODE
% -----------------------------------------------------------------------------------------------------------------------------------------------------------

\section{config/ApplicationConfig.java}
\lstsetjava
\begin{lstlisting}[language=Java, label={lst:ApplicationConfig}, caption=A required file in JavaEE for declaring the default application path when deployed.]
package com.partbuzz.slicer.config;


import javax.ws.rs.ApplicationPath;
import javax.ws.rs.core.Application;

/**
 * Created by mike on 1/9/16.
 */
@ApplicationPath("")
public class ApplicationConfig extends Application {
}
\end{lstlisting}


\section{config/CORSFilter}
\lstsetjava
\begin{lstlisting}[language=Java, label={lst:CORSFilter}, caption=As CuraEngine allows cross origin responses a filter is requred to allow access by default in WildFly.]
package com.partbuzz.slicer.config;

import java.io.IOException;

import javax.ws.rs.container.ContainerRequestContext;
import javax.ws.rs.container.ContainerResponseContext;
import javax.ws.rs.container.ContainerResponseFilter;
import javax.ws.rs.ext.Provider;

/**
 * @author mike
 */
@Provider
public class CORSFilter implements ContainerResponseFilter {

    @Override
    public void filter(final ContainerRequestContext requestContext,
                       final ContainerResponseContext cres) throws IOException {
        cres.getHeaders().add("Access-Control-Allow-Origin", "*");
        cres.getHeaders().add("Access-Control-Allow-Headers", "origin, content-type, accept, authorization");
        cres.getHeaders().add("Access-Control-Allow-Credentials", "true");
        cres.getHeaders().add("Access-Control-Allow-Methods", "GET, POST, PUT, DELETE, OPTIONS, HEAD");
        cres.getHeaders().add("Access-Control-Max-Age", "1209600");
    }

}
\end{lstlisting}

% -----------------------------------------------------------------------------------------------------------------------------------------------------------
% CURA CODE
% -----------------------------------------------------------------------------------------------------------------------------------------------------------

\section{cura/CuraEngine.java}
\lstsetjava
\begin{lstlisting}[language=Java, label={lst:CuraEngine}, caption=CuraEngine.java is the main code that preforms slicing throgh the CuraEngine C++ executable.]
/*
 * Copyright (c) 2016 Michael Meding -- All Rights Reserved.
 */
package com.partbuzz.slicer.cura;

import com.partbuzz.slicer.util.CuraEngineException;
import com.partbuzz.slicer.util.PlatformExecutable;

import java.io.IOException;
import java.io.InputStream;
import java.util.ArrayList;
import java.util.List;
import java.util.logging.Logger;

/**
 * Invoke the CURA slicing engine.
 * <p>
 * SAMPLE BASH
 * ../../CuraEngine-master/build/CuraEngine slice -v -j ultimaker2.json -g -e -o "output/test.gcode" -l ../../models/ControlPanel.stl
 *
 * @author mike
 */
public class CuraEngine extends PlatformExecutable {

    private static final Logger log = Logger.getLogger(CuraEngine.class.getName());
    private static final String CURAENGINE_PROG = "/usr/local/bin/CuraEngine";
    private final CuraEngineOptions options;

    public CuraEngine() {
        this.options = new CuraEngineOptions();
    }

    public CuraEngineOptions options() {
        return options;
    }

    public String output = "";

    /**
     * Setup a stream drainer.
     *
     * @param fp the input stream
     * @return the stream drainer
     * @throws java.io.IOException
     */
    protected static StreamDrainer setupStreamDrain(InputStream fp) throws IOException {
        final StreamDrainer drainer = new StreamDrainer(fp);
        // wait until we are draining
        synchronized (drainer) {

            log.info("starting thread");
            Thread t = new Thread(drainer);
            t.setDaemon(true);
            t.start();

            while (!drainer.hasStarted()) {
                try {
                    log.info("waiting for drainer thread");
                    drainer.wait(500);
                    log.info("drainer wait expired");
                } catch (InterruptedException ex) {
                    Thread.interrupted();
                    throw new IOException("Unable to setup stream drainer");
                }
            }

            log.info("drainer started");
        }

        return drainer;
    }

    /**
     * Run the CURA engine.
     *
     * @throws CuraEngineException
     */
    public void execute() throws IOException {
        checkPlatformExecutable(CURAENGINE_PROG);

        List<String> arguments = new ArrayList<>();
        arguments.add(CURAENGINE_PROG); // add base executable
        arguments.addAll(options.getOptions()); // add all options in the correct order

        for (String arg : arguments) {
            log.info("cura argument: " + arg);
        }

        // Process builder options for command line execution
        ProcessBuilder pb = new ProcessBuilder();
//		pb.directory(basePath);
        pb.redirectErrorStream(true);
        pb.command(arguments);

        // Create new thread and start execution
        Process p = pb.start();
        try {

            // Setup drains for both output and error streams
            StreamDrainer stdout = setupStreamDrain(p.getInputStream());
//            StreamDrainer stderr = setupStreamDrain(p.getErrorStream());

            int status = p.waitFor(); // wait for slice to finish

            if (status == 0) {

                // gather everything that is sent back from the command.
                StringBuilder sb = new StringBuilder();
                InputStream fp = p.getInputStream();
                byte[] buffer = new byte[512];
                while (fp.available() > 0) {
                    int n = fp.read(buffer);
                    sb.append(new String(buffer, 0, n));
                }
                output = sb.toString();
                log.info(output); // just log the output for now
                log.info("STDOUT:" + stdout.getText());
                return;
            } else {
                throw new CuraEngineException("unable to parse gcode file");
            }

//            if (options.haveIgnoreErrors()) {
//                log.warning(sb.toString());
//            } else {
//                throw new CuraEngineException(sb.toString());
//            }


        } catch (InterruptedException e) {
            throw new IOException(e);
        } finally {
            cleanupResources(p);
        }
    }

    /**
     * After execute is called the standard out of the call is concatenated into output.
     *
     * @return
     */
    public String getOutput() {
        return output;
    }
}
\end{lstlisting}


\section{cura/CuraEngineOptions.java}
\lstsetjava
\begin{lstlisting}[language=Java, label={lst:CuraEngineOptions}, caption=All arguments to the CuraEngine C++ executeable must be compiled before it can be executed. This code is the compiler for that argument list.]
package com.partbuzz.slicer.cura;

import com.partbuzz.slicer.util.CuraEngineException;

import java.util.ArrayList;
import java.util.List;

/**
 * Created by mike on 3/30/16.
 */

public class CuraEngineOptions {

    private String settingsFileName;
    private String outputFileName;
    private String modelFileName;
    private boolean verbose = false;
    private boolean p_option = false;
    private boolean g_option = false;
    private boolean e_option = false;
    private boolean ignoreErrors = false;

    /**
     * Set the verbose mode.
     *
     * @return the options
     */
    public CuraEngineOptions verbose() {
        this.verbose = true;
        return this;
    }

    /**
     * Set the settings file. This is a JSON formatted file which includes all needed information for slicing the file.
     *
     * @param filename the settings filename
     * @return the options
     */
    public CuraEngineOptions settingsFilename(String filename) {
        this.settingsFileName = filename;
        return this;
    }

    /**
     * Set the output filename. The resulting gcode file name/location.
     *
     * @param filename the output filename
     * @return the options
     */
    public CuraEngineOptions outputFilename(String filename) {
        this.outputFileName = filename;
        return this;
    }

    /**
     * Set the model filename. This is the model to be sliced.
     *
     * @param filename
     * @return options
     */
    public CuraEngineOptions modelFilename(String filename) {
        this.modelFileName = filename;
        return this;
    }

    /**
     * This option is similar to verbose but instead of just logging the output to the screen it logs the output to the CuraEngine log files.
     *
     * @return the options
     */
    public CuraEngineOptions logProgress() {
        this.p_option = true;
        return this;
    }

    /**
     * Switch setting focus to the current mesh group only.
     * Used for one-at-a-time printing.
     *
     * @return the options
     */
    public CuraEngineOptions currentGroupOnly() {
        g_option = true;
        return this;
    }

    /**
     * Adds a new extruder train for multi extrusion. For every time -e is included another extruder is added as on option to the slicer.
     *
     * @return the options
     */
    public CuraEngineOptions extruderTrainOption() {
        e_option = true;
        return this;
    }

    /**
     * For debugging.
     *
     * @return the options
     */
    public CuraEngineOptions ignoreErrors() {
        this.ignoreErrors = true;
        return this;
    }

    public boolean haveIgnoreErrors() {
        return ignoreErrors;
    }

    /*
     * Get the options in the right sequence. To build a correct CuraEngine command line executable.
     */
    public List<String> getOptions() throws CuraEngineException {
        List<String> list = new ArrayList<>();
        list.add("slice"); // the main parameter of the CuraEngine executable.
        if (verbose) {
            list.add("-v");
        }
        if (p_option) {
            list.add("-p");
        }

        if (settingsFileName == null) {
            throw new CuraEngineException("no setting file defined");
        } else {
            list.add("-j");
            list.add(settingsFileName);
        }
        if (g_option) {
            list.add("-g");
        }
        if (e_option) {
            list.add("-e");
        }
        if (outputFileName == null) {
            throw new CuraEngineException("no output file defined");
        } else {
            list.add("-o");
            list.add(outputFileName);
        }
        if (modelFileName == null) {
            throw new CuraEngineException("no model file defined");
        } else {
            list.add("-l");
            list.add(modelFileName);
        }

        return list;
    }
}
\end{lstlisting}


\section{cura/StreamDrainer.java}
\lstsetjava
\begin{lstlisting}[language=Java, label={lst:StreamDrainer}, caption=For ProcessBuilder to run correctly its output stream must be read from. This code spawns a separate thread to read and log this output stream.]
package com.partbuzz.slicer.cura;

import java.io.IOException;
import java.io.InputStream;
import java.util.logging.Level;
import java.util.logging.Logger;

/**
 * Drain an input stream into a buffer.
 */
class StreamDrainer implements Runnable {

    private static final Logger log = Logger.getLogger(StreamDrainer.class.getName());
    private final InputStream fp;
    private final StringBuilder sb;
    private boolean started;

    public StreamDrainer(InputStream fp) {
        this.fp = fp;
        this.sb = new StringBuilder();
        this.started = false;
    }

    public String getText() {
        return sb.toString();
    }

    boolean hasStarted() {
        return started;
    }

    @Override
    public void run() {

        // tell the invoker that we are draining
        log.info("starting drainer thread");
        synchronized (this) {
            log.info("drainer thread  sync");

            started = true;
            notifyAll();

            byte[] buffer = new byte[512];
            try {
                boolean reading = true;
                int n;
                do {
                    n = fp.read(buffer);
                    if (n > 0) {
                        sb.append(new String(buffer, 0, n));
                    }
                } while (n >= 0);
            } catch (IOException ex) {
                Logger.getLogger("PlatformExecutor").log(Level.SEVERE, null, ex);
            }
            log.info("done with the drainer");
        }
    }
}
\end{lstlisting}

% -----------------------------------------------------------------------------------------------------------------------------------------------------------
% REST CODE
% -----------------------------------------------------------------------------------------------------------------------------------------------------------

\section{rest/SlicerAPI.java}
\lstsetjava
\begin{lstlisting}[language=Java, label={lst:SlicerAPI}, caption=The Wildfly application server hosts a public RESTful API which is where requests for slicing and data processing get sent to their corresponding methods.]
package com.partbuzz.slicer.rest;

import com.partbuzz.slicer.cura.CuraEngine;
import com.partbuzz.slicer.util.CuraEngineException;
import com.partbuzz.slicer.util.FileTracker;
import os.io.FileHelper;
import os.util.ExceptionHelper;
import os.util.StringUtils;
import os.util.json.DefaultJSONFactory;
import os.util.json.JSONException;
import os.util.json.JSONObject;

import javax.mail.internet.MimeBodyPart;
import javax.mail.internet.MimeMultipart;
import javax.mail.util.ByteArrayDataSource;
import javax.servlet.http.HttpServletRequest;
import javax.ws.rs.*;
import javax.ws.rs.core.Context;
import javax.ws.rs.core.MediaType;
import javax.ws.rs.core.Response;
import java.io.*;
import java.util.HashMap;
import java.util.logging.Level;
import java.util.logging.Logger;

/**
 * Created by mike on 1/9/16.
 */
@Path("slicer")
public class SlicerAPI {
    private static final Logger log = Logger.getLogger(SlicerAPI.class.getName());
    private final DefaultJSONFactory JSONFactory = DefaultJSONFactory.getInstance();

    /**
     * Simply returns 200 OK if called.
     * <p>
     *
     * @param req
     * @return
     */
    @GET
    @Path("ping")
    @Produces({MediaType.TEXT_PLAIN})
    public Response ping(@Context HttpServletRequest req
    ) {
        log.info("ping called");
        return Response.ok("PONG", MediaType.TEXT_PLAIN).build();
    }

    /**
     * Add a new client to our database and file structure.
     * This will set aside all of the files needed for a new client and link them correctly
     *
     * @param req
     * @return The new uuid for that client so that we may track files
     */
    @POST
    @Path("setupClient")
    public Response setupClient(@Context HttpServletRequest req) {
        try {
            String uuid = FileTracker.setupNewClient();
            JSONObject payload = new JSONObject();
            payload.put("clientId", uuid);

            return Response.ok().entity(payload).build();
        } catch (CuraEngineException e) {
            e.printStackTrace();
            return Response.serverError().build();
        }
    }


    /**
     * Import an STL model file from the client
     *
     * @param req is the http request
     * @return the names of the uploaded files or an error message
     */
    @POST
    @Path("importStl/{clientId}")
    public Response importStl(@Context HttpServletRequest req,
                              @PathParam("clientId") String clientId) {
        try {

            ByteArrayDataSource bads = new ByteArrayDataSource(req.getInputStream(), req.getContentType());
            MimeMultipart mmp = new MimeMultipart(bads);

            // Read the raw file
            String filename = "";
            for (int i = 0; i < mmp.getCount(); i++) {
                MimeBodyPart mbp = (MimeBodyPart) mmp.getBodyPart(i);

                byte[] buffer = new byte[mbp.getSize()];
                mbp.getInputStream().read(buffer);

                // calculate a filename
//                String md5Hex = StringUtils.encodeHexString(StringUtils.md5(buffer));
//                filename = StringUtils.encode(md5Hex + "-" + mbp.getFileName());
                filename = StringUtils.encode(mbp.getFileName());

                // Save the file to disk
                File file = new File(FileTracker.getModelPathById(clientId) + FileTracker.delimiter + filename);
                FileOutputStream fp = new FileOutputStream(file);
                fp.write(buffer);
                fp.close();

            }

            // register our new model file
            String fileId = FileTracker.registerModelFile(clientId, filename);

            // build response message
            JSONObject response = new JSONObject();
            response.put("fileId", fileId);

            return Response.ok().entity(response.toString()).build();
        } catch (Exception e) {
            return Response.ok().entity(ExceptionHelper.getStackTrace(e)).build();
        }
    }

    /**
     * Import a settings file from the client
     *
     * @param req is the http request
     * @return the names of the uploaded files or an error message
     */
    @POST
    @Path("importSettings/{clientId}")
    public Response importSettings(@Context HttpServletRequest req,
                                   @PathParam("clientId") String clientId) {
        log.log(Level.INFO, "importing settings ");
        StringBuilder sb = new StringBuilder();
        try {
            String line;

            // parse input stream into a string
            try (BufferedReader br = new BufferedReader(new InputStreamReader(req.getInputStream()))) {
                while ((line = br.readLine()) != null) {
                    sb.append(line);
                }
            }
            String data = sb.toString(); // finish parse

            // Save to file
            String settingsPath = FileTracker.getSettingsFullPath(clientId);

            // write/overwrite file
            try (PrintWriter out = new PrintWriter(settingsPath)) {
                out.print(data);
            }

            return Response.ok().build();
        } catch (IOException e) {
            e.printStackTrace();
            return Response.serverError().entity("File write error").build();
        }
    }

    /**
     * Slice is the main function of this API. Given the parameters below it will return a formatted .gcode file
     * to the user.
     * <p>
     * The parameters required are a .stl file ID and a .json settings file ID.
     * These parameters are given when the above API methods are called. It is up to the client to track these ID's
     * <p>
     * SAMPLE JSON FORMAT
     * {"modelId":"UUID","settingsId":"UUID"}
     *
     * @param req is the http request
     * @return the names of the uploaded files or an error message
     */
    @POST
    @Path("slice/{clientId}/{modelId}")
    public Response slice(@Context HttpServletRequest req,
                          @PathParam("clientId") String clientId, @PathParam("modelId") String modelId) {
        try {
            // SAMPLE COMMAND RESULT
            // ../../CuraEngine-master/build/CuraEngine slice -v -j ultimaker2.json -g -e -o "output/test.gcode" -l ../../models/ControlPanel.stl

            // create new CuraEngine platform executable
            CuraEngine ce = new CuraEngine();
            ce.options() // add options. order is irrelevant.
                    .verbose()
                    .currentGroupOnly()
                    .extruderTrainOption()
                    .logProgress()
                    .settingsFilename(FileTracker.getSettingsFullPath(clientId))
                    .outputFilename(FileTracker.getOutputFilePath(clientId))
                    .modelFilename(FileTracker.getModelFullPath(clientId, modelId));
            ce.execute();

            // build response from output file and return it.
            JSONObject response = new JSONObject();
            byte[] raw = FileHelper.readContent(new File(FileTracker.getOutputFilePath(clientId)));
            String gcode = new String(raw);
            response.put("gcode", gcode);

            return Response.ok().entity(response.toString()).build();
        } catch (IOException e) {
            return Response.serverError().entity(e.getMessage()).build();
        }
    }

    /**
     * This function is to run a full test slice with a sample model and return its gcode.
     * Eliminates the need for a UI to run backend tests.
     * @param req is the http request
     * @return the names of the uploaded files or an error message
     */
    @POST
    @Path("testSlice")
    public Response testSlice(@Context HttpServletRequest req) {
        try {
            // create new CuraEngine platform executable
            CuraEngine ce = new CuraEngine();
            ce.options() // add options. order is irrelevant.
                    .verbose()
                    .currentGroupOnly()
                    .extruderTrainOption()
                    .logProgress()
                    .settingsFilename("/tmp/webslicer/test/prusa_i3.json")
                    .outputFilename("/tmp/webslicer/test/output.gcode")
                    .modelFilename("/tmp/webslicer/test/testModel.stl");
            ce.execute();

            // build response from output file and return it.
            JSONObject response = new JSONObject();
            byte[] raw = FileHelper.readContent(new File("/tmp/webslicer/test/output.gcode"));
            String gcode = new String(raw);
            response.put("gcode", gcode);

            return Response.ok().entity(response.toString()).build();
        } catch (IOException e) {
            return Response.serverError().entity(e.getMessage()).build();
        }
    }

    /**
     * Get all of the model file names and tracking ids associated with a client uuid.
     *
     * @param req
     * @param clientId
     * @return
     */
    @GET
    @Path("getFiles/{clientId}")
    @Produces("application/json")
    public Response getFiles(@Context HttpServletRequest req,
                             @PathParam("clientId") String clientId) {
        HashMap filesMap = FileTracker.getAllModelFiles(clientId);
        return Response.ok().entity(filesMap).build();
    }

    /**
     * A util function to parse an input JSON stream and parses to a JSONObject
     *
     * @param req
     * @return
     */
    private JSONObject parseJSONStream(HttpServletRequest req) {
        StringBuilder sb = new StringBuilder();
        try {
            String line;

            // parse input stream into a string
            try (BufferedReader br = new BufferedReader(new InputStreamReader(req.getInputStream()))) {
                while ((line = br.readLine()) != null) {
                    sb.append(line);
                }
            }
            String data = sb.toString();

            log.info(data);

            return JSONFactory.jsonObject(data);

        } catch (IOException | JSONException e) {
            log.severe(e.getMessage());
            throw new JSONException("Stream parse error.");
        }

    }
}
\end{lstlisting}

% -----------------------------------------------------------------------------------------------------------------------------------------------------------
% UTIL CODE
% -----------------------------------------------------------------------------------------------------------------------------------------------------------

\section{util/CuraEngineException.java}
\lstsetjava
\begin{lstlisting}[language=Java, label={lst:CuraEngineException}, caption=This exception class gets thrown when any issue arises while CuraEngine is preforming a slice.]
/*
 * Copyright (c) 2016 Michael Meding -- All Rights Reserved.
 */
package com.partbuzz.slicer.util;

import java.io.IOException;

/**
 * A CURA engine exception.
 *
 * @author mike
 */
public class CuraEngineException extends IOException {

	public CuraEngineException() {
	}

	public CuraEngineException(String message) {
		super(message);
	}

	public CuraEngineException(String message, Throwable cause) {
		super(message, cause);
	}

	public CuraEngineException(Throwable cause) {
		super(cause);
	}

}
\end{lstlisting}

\section{util/FileTracker.java}
\lstsetjava
\begin{lstlisting}[language=Java, label={lst:FileTracker}, caption=FileTracker tracks all of the files uploaded to the server to a series of hash maps. It also issues unique identifiers for each file that gets uploaded so that they may be tracked and accessed simply.]
package com.partbuzz.slicer.util;

import os.io.FileHelper;

import java.io.File;
import java.io.IOException;
import java.nio.file.Files;
import java.nio.file.Path;
import java.nio.file.Paths;
import java.util.HashMap;
import java.util.Map;
import java.util.UUID;

/**
 * For tracking the associated files with a particular client id.
 * FILE STRUCTURE
 * /main---/(clientUUID)---output.gcode
 * -settings.json
 * -/models---(uuid-filename.stl)
 * -(...stl)
 * -...
 * <p>
 * -/common---fdmprinter.json
 * -/presets---ultimaker.json
 * -prusai3.json
 * -...
 * <p>
 * Structure dictates file limit and will throw exceptions if # of files goes above limit.
 * <p>
 * Created by mike on 1/14/16.
 */
public class FileTracker {
    //TODO: This really needs to come from a properties file (JNDI path to properties file)
    public static String delimiter = File.separator;
    public static String basePath = delimiter + "tmp" + delimiter + "webslicer";
    public static String common = basePath + delimiter + "common";
    public static String modelDir = "models";
    public static String settingsFileName = "settings.json";
    public static String fdmprinterFile = "fdmprinter.json";
    public static String outputFile = "output.gcode";
    private static Map<String, HashMap<String, String>> clientFilesMap = new HashMap<>();

    /**
     * Same as above only for the model file.
     *
     * @param clientId The id of the client in question
     * @param fileId   The tracked filename which must be managed by client
     * @return
     */
    public static String getModelFileName(String clientId, String fileId) {
        return clientFilesMap.get(clientId).get(fileId);
    }

    /**
     * Get the full path to a model file
     *
     * @param clientId
     * @param fileId
     * @return
     */
    public static String getModelFullPath(String clientId, String fileId) {
        return basePath + delimiter + clientId + delimiter + modelDir + delimiter + getModelFileName(clientId, fileId);
    }

    /**
     * Get the path to the model file directory
     *
     * @return
     */
    public static String getModelPathById(String clientId) {
        return basePath + delimiter + clientId + delimiter + modelDir;
    }

    /**
     * Return the entire hashmap of model files associated with a client
     *
     * @param clientId
     * @return
     */
    public static HashMap<String, String> getAllModelFiles(String clientId) {
        return clientFilesMap.get(clientId);
    }

    public static String getOutputFilePath(String clientId) {
        return basePath + delimiter + clientId + delimiter + outputFile;
    }

    /**
     * Register a new model file with our registry
     *
     * @param fileName
     * @return
     */
    public static String registerModelFile(String clientId, String fileName) {
        String uuid = UUID.randomUUID().toString(); // generate new tracking uuid
        clientFilesMap.get(clientId).put(uuid, fileName); // store this filename using that new tracking uuid for that client
        return uuid;
    }

    /**
     * Gets the fully qualified path to the settings file for a specified client
     *
     * @param clientId
     * @return
     */
    public static String getSettingsFullPath(String clientId) {
        return basePath + delimiter + clientId + delimiter + settingsFileName;
    }

    /**
     * Remove all files associated with a client and remove its hashmap tracking in our server
     *
     * @param clientId
     */
    public static void removeClient(String clientId) {
        try {
            FileHelper.delete(new File(basePath + delimiter + clientId));
            clientFilesMap.remove(clientId);
        } catch (IOException e) {
            e.printStackTrace();
        }
    }

    /**
     * Completely remove model file from our disk
     *
     * @param clientId
     * @param id
     */
    public static void removeModelFile(String clientId, String id) {
        try {
            FileHelper.delete(new File(getModelFullPath(clientId, id)));
            clientFilesMap.get(clientId).remove(id);
        } catch (IOException e) {
            e.printStackTrace();
        }
    }


    /**
     * Register and setup a new client with our registry.
     *
     * @return the new clients UUID.
     */
    public static String setupNewClient() throws CuraEngineException {

        // generate new client uuid
        String uuid = UUID.randomUUID().toString();

        // Create empty file as clients home directory
        if (!new File(basePath + delimiter + uuid + delimiter + modelDir).mkdirs()) {
            throw new CuraEngineException("Could not create file path correctly");
        }

        // Symbolically link fdmprinter.json from common to this clients home directory
        try {
            Path targetPath = Paths.get(basePath + delimiter + uuid + delimiter + fdmprinterFile);
            Path sourceFile = Paths.get(common + delimiter + fdmprinterFile);
            Files.createSymbolicLink(targetPath, sourceFile);
        } catch (IOException e) {
            e.printStackTrace();
        }

        // Give our new client a file tracking hashmap
        clientFilesMap.put(uuid, new HashMap<String, String>()); // initalize client with new empty hashmap

        return uuid;
    }

}
\end{lstlisting}

\section{util/PlatformExecutable.java}
\lstsetjava
\begin{lstlisting}[language=Java, label={lst:PlatformExecutable}, caption=The PlatformExecutable class verifies if the local C++ executable is available and sets up the input and output pipes for CuraEngine if it is.]
/*
 * Copyright (c) 2016 Michael Meding -- All Right Reserved.
 */
package com.partbuzz.slicer.util;

import java.io.Closeable;
import java.io.File;
import java.io.IOException;
import java.io.InputStream;

/**
 * See if an executable is present and runnable.
 *
 * @author mike
 */
public abstract class PlatformExecutable {

	/**
	 * Check if a particular executable exists
	 *
	 * @param path is the path
	 */
	protected static void checkPlatformExecutable(String path) {
		File file = new File(path);
		if (!(file.exists() && file.canExecute())) {
			throw new UnsupportedOperationException(path + " is not supported on this platform");
		}
	}

	/**
	 * See if a port is in the valid range 0-65535
	 *
	 * @param port the port number
	 */
	protected static void checkPortNumberLegal(int port) {
		if (port < 0 || port > 65535) {
			throw new IllegalArgumentException("TCP/IP Port " + port + " illegal");
		}
	}

	/**
	 * Gather all the output a command may have sent back. Used if a system
	 * command did not finish successfully.
	 *
	 * @param p is the process
	 * @return the string
	 */
	protected static String gatherCommandOutput(Process p) throws IOException {
		StringBuilder sb = new StringBuilder();
		InputStream fp = p.getInputStream();
		byte[] buffer = new byte[512];
		while (fp.available() > 0) {
			int n = fp.read(buffer);
			sb.append(new String(buffer, 0, n));
		}

		return sb.toString();
	}

	/**
	 * Cleanup the external process resources
	 *
	 * @param p is the process
	 */
	protected static void cleanupResources(Process p) {
		if (p == null) {
			return;
		}

		close(p.getOutputStream());
		close(p.getInputStream());
		close(p.getErrorStream());
		p.destroy();
	}

	private static void close(Closeable c) {
		if (c != null) {
			try {
				c.close();
			} catch (Throwable t) {
				// ignore
			}
		}
	}
}
\end{lstlisting}