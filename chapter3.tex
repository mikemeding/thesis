\chapter{Methodology}

This chapter is mostly stubs and is heavily cut down. Origional content is left as context examples for \LaTeX mechanisms. 

\section{Subject Selection} 

All students read, understood, and signed a student assent form and their guardians read, understood, and signed a parental consent form (see Appendix~\ref{sec:parent_consent}). 

\section{Session Protocol}

At the conclusion of the activity, students were led in a group conversation about what they learned and developed. These conversations were also recorded. The questions asked by researchers during this conversation were variations of:
\begin{itemize}
\item What did you do to find your solution?
\item Did you do anything early on that you did again to help yourself?
\item If you had a friend who was coming in to work on this problem tomorrow,
what would you recommend they do?
\item Are there any other tricks you discovered?
\end{itemize}

%%%%%%%%%%%%%%%%%%%%%%%%%%%%%%%%%%%%%%%%%%%
%%%%%%%%%%%%%%%%%%%%%%%%%%%%%%%%%%%%%%%%%%%
\section{Design of Activities}
This is a beautiful, professonal-looking table.

\begin{table}
\begin{centering}
	\begin{tabular}{c  c   c}
	Week & Activity & Related Field \\ \hline
	1 & ``Rush Hour" Game & math problem solving \\ 
	2 & Light Optimization & electrical engineering \\ 
	3 & Gear Reduction & mechanical engineering \\ 
	4 & Word Search & computer science \\ 
	5 & Elevator Control & computer science \\ \hline
	\end{tabular}
	\caption[List of design activities.]{List of design activities, in order that they were conducted by students, with their related field.}
	\label{tab:activity-list}
\end{centering}
\end{table}

In total, the activities are intended to support computational thinking, as discussed in Chapter~\ref{chap:background}. Every activity represented a real engineering or design problem, had observable iteration behavior, defined metrics of success, and were process-oriented, as discussed in Section~\ref{sec:MEA}.

 \label{sec:success-levels}
Each activity had two levels of success. The first level is the completion of a successful design that solves the problem. The second level is synthesizing a general process for arriving at a solution. Both levels are defined as success criteria for each activity in the following sections. The five activities are now presented. They are also summarized in Table~\ref{tab:activity-list}.

%%%%%%%%%%%%%%%%%%%%%%%%%%%%%%%%%%%%%%%%%%%Rush Hour
\subsection{Parts of the study}

	I proceeded to have a subsection for each component activity of my study. Your study will likely be different in structure, so I removed it. I do have a few cool tables, so I left those in.
	
		\begin{table}
	\begin{centering}
	\begin{tabular}{l  l   l}
		\multicolumn{3}{l}{{\large Rush Hour Specifications}} \\
		\toprule
		\multirow{2}{*}{Success Criteria}  
			& Solution: 		& Solved puzzle \\ \cmidrule(r){2-3}
			& Process:		& Described Strategies \\ \midrule
		Iteration Metric	 & \multicolumn{2}{c}{Backtracking or resetting: deviation from current course }\\ 
		\bottomrule
	\end{tabular}
	\caption{Rush Hour activity success and iteration criteria.}
	\label{tab:spec-rush-hour}
	\end{centering}
	\end{table}
	
	\subsubsection{Purpose}
	


	\begin{table}
	\begin{centering}
	\begin{tabular}{l  l   l}
		\multicolumn{3}{l}{{\large Gear Reduction Specifications}} \\
		\toprule
		\multirow{2}{*}{Success Criteria}  
			& Solution: 		& Built transmission that lifts mass \\ \cmidrule(r){2-3}
			& Process:		& Non-trivial suggestions for a friend \\ \midrule
		Iteration Metric	 & \multicolumn{2}{c}{Applying energy to system, either by hand or by motor}\\ 
		\bottomrule
	\end{tabular}
	\caption{Gear Reduction activity success and iteration criteria.}
	\label{tab:spec-gears}
	\end{centering}
	\end{table}

%%%%%%%%%%%%%%%%%%%%%%%%%%%%%%%%%%%%%%%%%%%Word Search
\subsection{Week 4: Word Search}
	
	\begin{figure}
	\begin{centering}
	\begin{tabular}{|c|c|c|c|c|}
	\hline 
	Y & E & S & L & F\tabularnewline
	\hline
	\hline 
	M & N & H & R & I\tabularnewline
	\hline
	\hline 
	P & T & O & U & R\tabularnewline
	\hline
	\hline 
	A & C & E & M & E\tabularnewline
	\hline
	\end{tabular}
	\par\end{centering}
	
	\caption{Example word search puzzle provided to students.}
	\label{fig:Example-word-search}
	\end{figure}
	


%%%%%%%%%%%%%%%%%%%%%%%%%%%%%%%%%%%%%%%%%%%Analysis
%%%%%%%%%%%%%%%%%%%%%%%%%%%%%%%%%%%%%%%%%%%Plan
\section{Coding and Analysis} \label{sec:analysis-plan}
Videos from student sessions were analyzed using coding techniques informed by \citet{welch} and \citet{REESE}, where specific codes were defined to describe important behaviors. Additional codes were created in the style of grounded theory \citep{strauss97}: as behaviors and patterns that were not predicted were observed. 

The body of codes covered a wide range of design behaviors, as seen in Appendix~\ref{tab:starting-codes}, but only the TEST code was used in analysis, as it was used to characterize design iteration. 
