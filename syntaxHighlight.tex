\usepackage{listings}
\lstloadlanguages{Java,sh,bash,Haskell,HTML,PHP,XML}
\lstdefinelanguage{console}{
  morekeywords={},
  otherkeywords={warumgehtdasnicht>,\$}
}

%----------------------------------------------------------------------------------------
%	COLORS
%----------------------------------------------------------------------------------------

\definecolor{dkgreen}{rgb}{0,0.6,0}
\definecolor{mauve}{rgb}{0.58,0,0.82}
\definecolor{ULgray}{HTML}{ebebeb}

\colorlet{punct}{red!60!black}
 \definecolor{background}{HTML}{EEEEEE}
 \definecolor{delim}{RGB}{20,105,176}
 \colorlet{numb}{magenta!60!black}

 \definecolor{editorLightGray}{cmyk}{0.05, 0.05, 0.05, 0.1}
 \definecolor{editorGray}{cmyk}{0.6, 0.55, 0.55, 0.2}
 \definecolor{editorPurple}{cmyk}{0.5, 1, 0, 0}
 \definecolor{editorWhite}{cmyk}{0, 0, 0, 0}
 \definecolor{editorBlack}{cmyk}{1, 1, 1, 1}
 \definecolor{editorOrange}{cmyk}{0, 0.8, 1, 0}
 \definecolor{editorBlue}{cmyk}{1, 0.6, 0, 0}
 \definecolor{editorPink}{cmyk}{0, 1, 0, 0}


%--------------------------------------------BASH-----------------------------------------------------

\newcommand{\lstsetconsole}
{ \lstset{%language=sh,
        lineskip=-2pt,
        breaklines=true,
        language=console,
        breaklines=true,
        commentstyle=\textit,
        keywordstyle=\bfseries,
        basicstyle=\ttfamily,
        stringstyle=\ttfamily,
        showstringspaces=false,
        frame=single,
        tabsize=2
  }
}
\lstdefinelanguage{scalaconsole}{
  morekeywords={},
  otherkeywords={scala>,\|}
}

%--------------------------------------------BASH 2-----------------------------------------------------
\newcommand{\lstsetrepl}
{ \lstset{%language=sh,
        lineskip=-2pt,
        breaklines=true,
        language=scalaconsole,
        breaklines=true,
        commentstyle=\textit,
        keywordstyle=\bfseries,
        basicstyle=\ttfamily,
        stringstyle=\ttfamily,
        showstringspaces=false,
        frame=single,
        tabsize=2
  }
}

%--------------------------------------------JAVA-----------------------------------------------------
\newcommand{\lstsetjava}{
\lstset{
		 language=Java,
         basicstyle=\footnotesize\color{red}\ttfamily, % standard font
         numbers=left,               % line number position
         numberstyle=\tiny,          % line number style
         %stepnumber=2,               % distance between lines
         % line number settings
         numbersep=5pt,              % distance between number + text
         tabsize=2,                  % tab size
         extendedchars=true,         %
         breaklines=true,            % line breaks
		 keywordstyle=\color{cyan},
		 identifierstyle=\color{Blue},
         stringstyle=\color{BurntOrange}\ttfamily, % String color
         showspaces=false,           % show spaces?
         showtabs=false,             % show tabs?
         xleftmargin=17pt,
         frame=none,
         framexleftmargin=17pt,
         framexrightmargin=5pt,
         framexbottommargin=4pt,
%         backgroundcolor=\color{ULgray},
         showstringspaces=false      % Spaces in strings?
 }
}

\lstdefinelanguage{Java}
{
 morecomment = [l]{//},
 morecomment = [s]{/*}{*/},
 morestring=[b]",
 sensitive = false,
 morekeywords = {abstract,  event,  new,  struct,
  as,  explicit,  null,  switch,
  base,  extern,  object,  this,
  bool,  false,  operator,  throw,
  break,  finally,  true,
  byte,  fixed,  override,  try,
  case,  float,  params,  typeof,
  catch,  for,  private,  uint,
  char,  foreach,  protected,  ulong,
  checked,  goto,  public,  unchecked,
  class,  if,  readonly,  unsafe,
  const,  implicit,  ref,  ushort,
  continue,  in,  return,  using,
  decimal,  int,  sbyte,  virtual,
  default,  interface,  sealed,  volatile,
  delegate,  internal,  short,  void,
  do,  is,  sizeof,  while,
  double,  lock,  stackalloc,  
  else,  long,  static,  
  enum,  namespace,  String, implements}
}

%--------------------------------------------SCALA-----------------------------------------------------
\lstdefinelanguage{scala}{
  morekeywords={abstract,case,catch,class,def,%
    do,else,extends,false,final,finally,%
    for,forSome,if,implicit,import,lazy,match,mixin,%
    new,null,object,override,package,%
    private,protected,requires,return,sealed,%
    super,this,throw,trait,true,try,%
    type,val,var,while,with,yield},
  otherkeywords={_,:,=,=>,<-,<\%,<:,>:,\#,@},
  sensitive=true,
  morecomment=[l]{//},
  morecomment=[n]{/*}{*/},
  morestring=[b]",
  morestring=[b]',
  morestring=[b]"""
}
\newcommand{\lstsetscala}{
 \lstset{language=scala,
        breaklines=true,
        commentstyle=\textit,
        keywordstyle=\bfseries,
        basicstyle=\ttfamily,
        stringstyle=\ttfamily,
        showstringspaces=false,
        frame=single,
        tabsize=2
        %%linewidth=\textwidth,captionpos=b
        %numbers=left, stepnumber=5, numbersep=10pt
 }
}


%--------------------------------------------HTML-----------------------------------------------------
\newcommand{\lstsethtml}{
 \lstset{language=HTML,
        breaklines=true,
        commentstyle=\textit,
        keywordstyle=\bfseries,
        basicstyle=\ttfamily,
        stringstyle=\ttfamily,
        showstringspaces=false,
        frame=single,
        tabsize=2
        %%linewidth=\textwidth,captionpos=b
        %numbers=left, stepnumber=5, numbersep=10pt
 }
}


%--------------------------------------------PHP-----------------------------------------------------
\newcommand{\lstsetphp}{
 \lstset{language=PHP,
        breaklines=true,
        commentstyle=\textit,
        keywordstyle=\bfseries,
        basicstyle=\ttfamily,
        stringstyle=\ttfamily,
        showstringspaces=false,
        frame=single,
        tabsize=2
        %%linewidth=\textwidth,captionpos=b
        %numbers=left, stepnumber=5, numbersep=10pt
 }
}


%--------------------------------------------HASKELL-----------------------------------------------------
\lstnewenvironment{code}
    {\lstset{}%
      \csname lst@SetFirstLabel\endcsname}
    {\csname lst@SaveFirstLabel\endcsname}
\newcommand{\lstsethaskell}{
    \lstset{
      language=Haskell,
      commentstyle=\textit,
      keywordstyle=\bfseries,
      basicstyle=\ttfamily,
      stringstyle=\ttfamily,
      showstringspaces=false,
      frame=single,
      flexiblecolumns=false,
      basewidth={0.5em,0.45em},
      literate={+}{{$+$}}1 {/}{{$/$}}1 {*}{{$*$}}1 {=}{{$=$}}1
               {==}{{$==$}}2 %{!=}{{$\not\equiv$}}2
               {>}{{$>$}}1 {<}{{$<$}}1 {\\}{{$\lambda$}}1
               {\\\\}{{\char`\\\char`\\}}1
               {->}{{$\rightarrow$} }2 {>=}{{$\geq$}}2 {<-}{{$\leftarrow$}}2
               {<=}{{$\leq$}}2 {=>}{{$\Rightarrow$} }2
               {\ .}{{$\circ$}}2 {\ .\ }{{$\circ$}}2 {(.)}{({$\circ$})}2
               {>>}{{>>}}2 {>>=}{{>>=}}2
               {|}{{$\mid$}}1
    }
}

%--------------------------------------------JAVASCRIPT-----------------------------------------------------
\lstdefinelanguage{JavaScript}{
  keywords={typeof, new, true, false, catch,%
    function, return, null, catch, switch, var,%
    if, in, while, do, else, case, break},
  ndkeywords={class, export, boolean, throw, implements, import, this},
  sensitive=false,
  comment=[l]{//},
  morecomment=[s]{/*}{*/},
  morestring=[b]',
  morestring=[b]"
}
\newcommand{\lstsetjavascript}{
  \lstset{
		% language=JavaScript,
		% breaklines=true,
		% commentstyle=\textit,
		% basicstyle=\ttfamily,
		% keywordstyle=\bfseries,
		% stringstyle=\ttfamily,
		% showstringspaces=false,
		% frame=single,
		% tabsize=2
    basicstyle=\footnotesize\color{red}\ttfamily, % standard font
         numbers=left,               % line number position
         numberstyle=\tiny,          % line number style
         %stepnumber=2,               % distance between lines
         % line number settings
         numbersep=5pt,              % distance between number + text
         tabsize=2,                  % tab size
         extendedchars=true,         %
         breaklines=true,            % line breaks
     keywordstyle=\color{cyan},
     identifierstyle=\color{Blue},
         stringstyle=\color{BurntOrange}\ttfamily, % String color
         showspaces=false,           % show spaces?
         showtabs=false,             % show tabs?
         xleftmargin=17pt,
         frame=none,
         framexleftmargin=17pt,
         framexrightmargin=5pt,
         framexbottommargin=4pt,
%         backgroundcolor=\color{ULgray},
         showstringspaces=false      % Space
  }
}

% ---------------------------------------- JSON -----------------------------------------------------
\lstdefinelanguage{json}{
    basicstyle=\normalfont\ttfamily,
    numbers=left,
    numberstyle=\scriptsize,
    stepnumber=1,
    numbersep=8pt,
    showstringspaces=false,
    breaklines=true,
    frame=lines,
    backgroundcolor=\color{background},
    literate=
     *{0}{{{\color{numb}0}}}{1}
      {1}{{{\color{numb}1}}}{1}
      {2}{{{\color{numb}2}}}{1}
      {3}{{{\color{numb}3}}}{1}
      {4}{{{\color{numb}4}}}{1}
      {5}{{{\color{numb}5}}}{1}
      {6}{{{\color{numb}6}}}{1}
      {7}{{{\color{numb}7}}}{1}
      {8}{{{\color{numb}8}}}{1}
      {9}{{{\color{numb}9}}}{1}
      {:}{{{\color{punct}{:}}}}{1}
      {,}{{{\color{punct}{,}}}}{1}
      {\{}{{{\color{delim}{\{}}}}{1}
      {\}}{{{\color{delim}{\}}}}}{1}
      {[}{{{\color{delim}{[}}}}{1}
      {]}{{{\color{delim}{]}}}}{1},
}

%-------------------------------------------XML-----------------------------------------------------
\newcommand{\lstsetxml}{
 \lstset{language=XML,
        breaklines=true,
        commentstyle=\sffamily\color{dkgreen},
        identifierstyle=\color{blue},
        keywordstyle=\bfseries\color{cyan},
        basicstyle=\footnotesize\sffamily,
        showstringspaces=false,
        stringstyle=\ttfamily\color{BurntOrange},
        frame=none,
        tabsize=2,
        literate=
        %linewidth=\textwidth,captionpos=b
        %numbers=left, stepnumber=5, numbersep=10pt
 }
 }
 \lstdefinelanguage{XML}
 {
   basicstyle=\ttfamily\color{blue},
   morestring=[s]{"}{"},
   morecomment=[s]{!--}{--},
   morekeywords={xmlns,version,type,name}% list your attributes here       
 } 

%--------------------------------------------CSHARP-----------------------------------------------------
\lstdefinelanguage{CSharp}{
 morekeywords = {abstract,event,new,struct,as,explicit,%
    null,switch,base,extern,object,this,bool,false,%
    operator,throw,break,finally,out,true,byte,fixed,%
    override,try,case,float,params,typeof,catch,for,%
    private,uint,char,foreach,protected,ulong,checked,%
    goto,public,unchecked,class,if,readonly,unsafe,%
    const,implicit,ref,ushort,continue,in,return,using,%
    decimal,int,sbyte,virtual,default,interface,sealed,%
    volatile,delegate,internal,short,void,do,is,sizeof,%
    while,double,lock,stackalloc,else,long,static,%
    enum,namespace,string,partial},
  morecomment = [l]{//},
  morecomment = [l]{///},
  morecomment = [s]{/*}{*/},
  morestring=[b]",
  sensitive = true
}
\newcommand{\lstsetcsharp}{
 \lstset{language=csharp,
        breaklines=true,
        commentstyle=\sffamily,
        basicstyle=\sffamily,
        keywordstyle=\bfseries,
        stringstyle=\ttfamily,
        showstringspaces=false,
        frame=single,
        tabsize=2
        %%linewidth=\textwidth,captionpos=b
        %numbers=left, stepnumber=5, numbersep=10pt
 }
}

%--------------------------------------------FSHARP-----------------------------------------------------
\lstdefinelanguage{FSharp}{
  morekeywords={abstract,and,as,assert,base,begin,%
    class,default,delegate,do,done,downcast,downto,%
    elif,else,end,exception,extern,false,finally,for,fun,%
    function,if,in,inherit,inline,interface,internal,lazy,%
    let,match,member,module,mutable,namespace,%
    new,not,null,of,open,or,override,private,public,rec,%
    return,static,struct,then,to,true,try,type,upcast,use,%
    val,void,when,while,with,yield,asr,land,lor,lsl,lsr,lxor,%
    mod,sig,atomic,break,checked,component,const,%
    constraint,constructor,continue,eager,event,external,%
    fixed,functor,global,include,method,mixin,object,%
    parallel,process,protected,pure,sealed,tailcall,trait,virtual,volatile},     
  sensitive=false,
  morecomment=[l][\color{greencomments}]{///},
  morecomment=[l][\color{greencomments}]{//},
  morecomment=[s][\color{greencomments}]{{(*}{*)}},
  morestring=[b]"
}
\newcommand{\lstsetfsharp}{
 \lstset{language=fsharp,
        breaklines=true,
        commentstyle=\sffamily,
        basicstyle=\sffamily,
        keywordstyle=\bfseries,
        stringstyle=\ttfamily,
        showstringspaces=false,
        frame=single,
        tabsize=2
        %%linewidth=\textwidth,captionpos=b
        %numbers=left, stepnumber=5, numbersep=10pt
 }
}
