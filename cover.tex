% Sherman 1
% Revision 1.1  92/04/22  13:08:20  epeisach

% BE SURE TO READ THE UNIVERSITY'S RULES ON WHAT FIELDS ARE REQUIRED OR ENCOURANGED FOR YOUR DEPARTMENT

\title{Implementing a Web-Based\\ Model Slicer for \\ 3D Printers }

\author{Michael U.B. Meding}

\prevdegrees{B.S., University of Massachusetts Lowell (2015)}

\department{Department of Computer Science}

% If the thesis is for two degrees simultaneously, list them both
% separated by \and like this:
% \degree{Doctor of Philosophy \and Master of Science}
\degree{Master of Science}
\degreemonth{December}
\degreeyear{2016}
\thesisdate{December 1, 2010}

% If there is more than one supervisor, use the \supervisor command
% once for each.
\supervisor{Fred G. Martin}{Associate Professor}
\reader{ Jeff Brown}{Associate Professor} % reader is not necessary

% this is the department committee chairman, not the thesis committee chairman. 
\chairman{Jie Wang}{Department Chairman}

% Make the titlepage based on the above information.  If you need
% something special and can't use the standard form, you can specify
% the exact text of the titlepage yourself.  Put it in a titlepage
% environment and leave blank lines where you want vertical space.
% The spaces will be adjusted to fill the entire page.  The dotted
% lines for the signatures are made with the \signature command.
\maketitle

% The abstractpage environment sets up everything on the page except
% the text itself.  The title and other header material are put at the
% top of the page, and the supervisors are listed at the bottom.  A
% new page is begun both before and after.  Of course, an abstract may
% be more than one page itself.  If you need more control over the
% format of the page, you can use the abstract environment, which puts
% the word "Abstract" at the beginning and single spaces its text.

%% You can either \input (*not* \include) your abstract file, or you can put
%% the text of the abstract directly between the \begin{abstractpage} and
%% \end{abstractpage} commands.

% First copy: start a new page, and save the page number.
\newpage
\thispagestyle{empty}
\mbox{}
\newpage
\thispagestyle{empty}

% Second copy: start a new page, and reset the page number.  This way,
% the second copy of the abstract is not counted as separate pages.
\pagestyle{plain}
\newpage
\setcounter{page}{1}
\pagenumbering{roman}
\begin{abstractpage}
% meding 1
% $Log: abstract.tex,v $
% Revision 1.1  93/05/14  14:56:25  starflt
% Initial revision
% 
% Revision 1.1  90/05/04  10:41:01  lwvanels
% Initial revision
% 
%
%% The text of your abstract and nothing else (other than comments) goes here.
%% It will be single-spaced and the rest of the text that is supposed to go on
%% the abstract page will be generated by the abstractpage environment.  This
%% file should be \input (not \include 'd) from cover.tex.

3D printing currently has a large gap between software and hardware. A hobbyist machine can now be purchased for less than \$500 but having good software to drive it is hard to find. Currently the only competent and free slicing software available is Cura and Repetier Host. Currently, Cura has varied support on all platforms, and Repetier Host is intended only for Windows. Neither software have any web support nor will they be likely to have any support in the future as the slicing process requires a computer with a considerable amount of both graphical and computational power. The purpose of this research is to construct a web based slicing software and make it simple for users without any prior knowledge of 3D printing to take full advantage of their printer and as a result will make 3D printing much more approachable for users who are not computer savvy. Additionally, this opens up opportunities for educators in STEM programs to teach students about 3D printing in a simple and practical way.

\end{abstractpage}

%%%%%%%%%%%%%%%%%%%%%%%%%%%%%%%%%%%%%%%%%%%%%%%%%%%%%%%%%%%%%%%%%%%%%%
% -*-latex-*-



