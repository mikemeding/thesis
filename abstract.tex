% meding 1
% $Log: abstract.tex,v $
% Revision 1.1  93/05/14  14:56:25  starflt
% Initial revision
% 
% Revision 1.1  90/05/04  10:41:01  lwvanels
% Initial revision
% 
%
%% The text of your abstract and nothing else (other than comments) goes here.
%% It will be single-spaced and the rest of the text that is supposed to go on
%% the abstract page will be generated by the abstractpage environment.  This
%% file should be \input (not \include 'd) from cover.tex.

3D printing currently has a large gap between software and hardware. A hobbyist machine can now be purchased for less than \$500 but having good software to drive it is hard to find. Currently the only competent and free slicing software available is Cura and Repetier Host. Currently, Cura has varied support on all platforms, and Repetier Host is intended only for Windows. Neither software have any web support nor will they be likely to have any support in the future as the slicing process requires a computer with a considerable amount of both graphical and computational power. The purpose of this research is to construct a web based slicing software and make it simple for users without any prior knowledge of 3D printing to take full advantage of their printer and as a result will make 3D printing much more approachable for users who are not computer savvy. Additionally, this opens up opportunities for educators in STEM programs to teach students about 3D printing in a simple and practical way.
