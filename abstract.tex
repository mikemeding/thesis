% abstract
In the field of 3D printing, there is currently a large gap between hardware and software.
A hobbyist machine can now be purchased for less than \$500, but optimal software can be difficult to find, and the only free options available are Cura and Repetier Host. 
Cura may be run on the Windows, Mac, and Linux operating systems, while Repetier Host is intended only for Windows.
This leaves a considerable software gap, which could be filled by a web-based application unbound to a particular system.
The purpose of this research was to construct an open source, web-based slicing software to facilitate the use of 3D printers, making 3D printing more approachable for users who have limited experience with technology. 
Additionally, this application creates opportunities for STEM educators to teach students about 3D printing in a simple and practical way, thereby beginning to bridge the gap between hardware and software.
