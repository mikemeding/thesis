\chapter{Client Side Code Appendix}
\paragraph{}
Included in this appendix is the full source code for the client side WebSlicer application.


\section{app.js}
\lstsetjavascript
\begin{lstlisting}[language=JavaScript, label={lst:app}, caption=Main app code where the angular module WebSlicer is defined. This is similar to a main function in C++.]
/**
 * Created by mike on 9/25/15.
 */
(function () {
    var app = angular.module("WebSlicer", ["ui.router", "ui.bootstrap", "ngAnimate", "ngFileUpload"]);

    app.controller("Main", ["$http", "$scope", "$window", "$rootScope", function ($http, $scope, $window, $rootScope) {

        // enable all bootstrap tooltips
        //$(function () {
        //    $('[data-toggle="tooltip"]').tooltip()
        //});

        // the main servers base url for making REST calls
        $rootScope.baseUrl = "http://localhost:8080/WebSlicer/slicer";
        $rootScope.clientId = "";
        $rootScope.busy = false;

        // environment vars
        //$rootScope.baseUrl;
        $scope.title = "Web Slicer";
        $scope.displayOctoprintData = false;
        //$scope.clientId = "";
        $scope.modelFiles = {};
        $scope.gcode = {"gcode": ""};

        // make gcode available for download
        var blob = new Blob([$scope.gcode.gcode], {type: 'text/plain'});
        $scope.gcodeDownload = (window.URL || window.webkitURL).createObjectURL(blob);

        /**
         * Generate a new client id and layout all of the files for this new client on the server
         */
        $scope.generateClientId = function () {
            $http({
                method: 'POST',
                url: $rootScope.baseUrl + "/setupClient"
            }).then(function successCallback(response) {
                console.log(response);
                $rootScope.clientId = response.data.clientId;
            }, function errorCallback(response) {
                console.error(response);
            });
        };

        //DOWNLOAD GCODE FUNCTION
        $scope.downloadGcode = function (gcode) {
            var blob = new Blob([gcode], {type: "application/json;charset=utf-8;"});
            var downloadLink = angular.element('<a></a>');
            //var downloadLink = angular.element(elem.querySelector(".downloadGcode"));
            downloadLink.attr('href', window.URL.createObjectURL(blob));
            downloadLink.attr('download', 'output.gcode');
            downloadLink[0].click();

            //$window.open("data:application/json;charset=utf-8," + encodeURIComponent(gcode));
        };


        /**
         * The main slice function. This will only operate if the client has both uploaded a settings file and a model file.
         * Both of these files will be given an id that the client must track in order to slice a file properly.
         *
         * @param modelId
         */
        $rootScope.slice = function (modelId) {
            if (modelId) {
                $rootScope.busy = true;

                // run import settings to update settings on server BEFORE SLICING
                $rootScope.importSettings()

                    .then(function successCallback(response) {
                        console.log(response);

                        // SEND SLICE REQUEST!
                        $http({
                            method: 'POST',
                            url: $rootScope.baseUrl + "/slice/" + $rootScope.clientId + "/" + modelId
                        }).then(function successCallback(response) {
                            // capture our gcode as a response.
                            console.log(response);
                            $scope.gcode = response.data;
                            $rootScope.busy = false;

                            // render our gcode in the 2D viewer
                            $rootScope.render(response.data.gcode);

                        }, function errorCallback(response) {
                            console.error(response);
                            $rootScope.busy = false;
                        });

                    }, function errorCallback(response) {
                        console.error(response);
                        console.error("import settings before slice");
                    });

            } else {
                console.error("model or settings id missing");
            }
        };
    }]);

    app.config(["$stateProvider", "$urlRouterProvider", "$controllerProvider", "$compileProvider", function appConfig($stateProvider, $urlRouterProvider, $controllerProvider, $compileProvider) {

        // default route
        $urlRouterProvider.otherwise("home");

        // states
        $stateProvider
            .state("home", {
                url: "/home",
                templateUrl: "public/index.html"
            });

        // controller config
        $controllerProvider.allowGlobals();

        // post compile config
        $compileProvider.aHrefSanitizationWhitelist(/^\s*(https?|ftp|mailto|tel|file|blob):/);
    }]);
    
    app.directive('fileModel', ['$parse', function ($parse) {
        return {
            restrict: 'A',
            link: function (scope, element, attrs) {
                var model = $parse(attrs.fileModel);
                var modelSetter = model.assign;

                element.bind('change', function () {
                    scope.$apply(function () {
                        modelSetter(scope, element[0].files[0]);
                    });
                });
            }
        };
    }]);
})();
\end{lstlisting}


\section{index.html}
\lstsetxml
\begin{lstlisting}[language=HTML, label={lst:index}, caption=As this is a single page web application this is the only HTML file. It contains all of the libraries needed to run all subsequent sections.]
<!DOCTYPE html>
<html ng-app="WebSlicer">
<head lang="en">
    <meta charset="UTF-8">
    <title>Web Slicer</title>

    <link rel="icon" type="image/png" href="img/logo22-plain.png"/>

    <!--libraries-->
    <script src="bower_components/jquery/dist/jquery.js"></script>
    <script src="bower_components/bootstrap/dist/js/bootstrap.js"></script>
    <script src="bower_components/angular/angular.js"></script>
    <script src="bower_components/angular-animate/angular-animate.js"></script>
    <script src="bower_components/angular-ui-router/release/angular-ui-router.js"></script>
    <script src="bower_components/angular-bootstrap/ui-bootstrap.js"></script>
    <script src="bower_components/angular-bootstrap/ui-bootstrap-tpls.js"></script>
    <script src="bower_components/ng-file-upload/ng-file-upload.js"></script>

    <link rel="stylesheet" href="bower_components/font-awesome/css/font-awesome.css">

    <!--css-->
    <!--<link rel="stylesheet" href="bower_components/bootstrap/dist/css/bootstrap.css">-->
    <link rel="stylesheet" href="bower_components/bootstrap/dist/css/bootstrap-cosmo.css">
    <link rel="stylesheet" href="css/custom.css"/>

    <!-- MAIN CONTROLLER AND CONFIG -->
    <script src="app.js"></script>

    <!--FILE CONTROLS-->
    <script src="model/files/filesController.js"></script>
    <script src="model/files/filesDirective.js"></script>

    <!--SETTINGS-->
    <script src="model/profile/settingsController.js"></script>
    <script src="model/profile/settingsTabDirective.js"></script>
    <script src="model/profile/settingsBuilderService.js"></script>

    <!--OCTOPRINT-->
    <script src="model/octoprint/octoprintService.js"></script>
    <script src="model/octoprint/octoprintDirective.js"></script>
    <script src="model/octoprint/octoprintController.js"></script>

    <!--VISUALIZER-->
    <script src="viewer/lib/zlib.min.js"></script>
    <script src="viewer/gcodeFactory.js"></script>
    <script src="viewer/renderFactory.js"></script>
    <script src="viewer/viewer.js"></script>

</head>
<body ng-controller="Main">

<div class="container" ng-controller="SettingsController">
<div class="row">
    <div class="col-lg-6 col-lg-offset-3">
        <h1 class="title"><img src="img/logo22.svg" width="150px">{{title}}</h1>
    </div>
</div>
<div class="row">
<div class="col-lg-6 col-lg-offset-3">
    <div class="panel panel-default">
        <div class="panel-heading">
            <h3>Login</h3>
        </div>
        <div class="panel-body">
        <form>
        <div class="form-group">
        <label for="clientId">Client ID</label>
        <input class="form-control" type="text" id="clientId" ng-model="$root.clientId">
        <button class="btn btn-warning" ng-click="generateClientId()">generate new id</button>
        </div>
        </form>
        </div>
    </div>
</div>
</div>

<div class="row">

<!--CONTROLS-->
<div class="col-lg-6">

<!--<filecontrols></filecontrols>-->
<div ng-controller="FilesController">

<div class="panel panel-primary">
    <div class="panel-heading">
        <h3>Slicer Controls</h3>
    </div>
    <div class="panel-body">
        <ul class="nav nav-pills nav-stacked">

            <!--file chooser-->
            <li role="presentation">

                <button class="btn btn-primary" type="file" ngf-select="uploadFile($file,$invalidFiles)"
                        ngf-max-height="1000" ngf-max-size="2MB">
                    Upload File
                </button>

                <div ng-show="f.progress >= 0">
                <h3>{{f.name}} {{errFile.name}} {{errFile.$error}} {{errFile.$errorParam}}</h3>
                <div class="progress">
                <div class="progress-bar" role="progressbar" aria-valuenow="{{f.progress}}"
                     aria-valuemin="0"
                     aria-valuemax="100" style="width: {{f.progress}}%;">
                    <!--<div class="progress-bar" role="progressbar" aria-valuenow="{{f.progress}}" aria-valuemin="0"-->
                    <!--aria-valuemax="100">-->
                    {{f.progress}}% Complete
                </div>
                </div>
                </div>

            </li>

            <!--current file list-->
            <li role="presentation" ng-if="$root.clientId">
                <p>{{modelFileId}}</p>

                <table class="table table-hover table-condensed">
                    <tbody>

                    <tr ng-repeat="item in modelFiles">
                    <td>{{item.value}}
                    <button ng-if="!$root.busy" class="btn btn-primary btn-sm pull-right"
                            ng-click="$root.slice(item.key)">
                        Slice
                    </button>
                    <i ng-if="$root.busy" class="fa fa-3x fa-cog fa-spin pull-right"></i>


                    </td>
                    </tr>

                    </tbody>
                </table>

            </li>

            <!--slice button-->
            <li role="presentation">
                <!--<button class="btn btn-primary" ng-click="getFileList()">Refresh file list</button>-->
            </li>

        </ul>

    </div>
</div>


    </div>
</div>

<!--<div class="col-lg-6">-->
<!--<octoprint></octoprint>-->
<!--</div>-->

</div>

<div class="row" ng-controller="SettingsController">

<!--SETTINGS PANEL-->
<div class="col-lg-6">

    <div class="panel panel-default">
        <div class="panel-heading">
            <h3>Printer Settings</h3>
        </div>
        <div class="panel-body">

            <uib-tabset>
            <uib-tab heading="basic">
                <curasettings src="model/profile/json/basic.json"></curasettings>
            </uib-tab>
            <uib-tab heading="advanced">
                <curasettings src="model/profile/json/advanced.json"></curasettings>
            </uib-tab>
            <uib-tab heading="fine-tune">
                <curasettings src="model/profile/json/fine-tune.json"></curasettings>
            </uib-tab>
            <uib-tab heading="machine">
                <curasettings src="model/profile/json/machine.json"></curasettings>
            </uib-tab>
            <uib-tab heading="start/end gcode">
                <curasettings src="model/profile/json/start-end-gcode.json"></curasettings>
            </uib-tab>
            </uib-tabset>

        </div>
    </div>

</div>

<!--GCODE OUTPUT PANEL-->
<div class="col-lg-6" ng-if="gcode.gcode">
    <div class="panel panel-success">
        <div class="panel-heading">
            <h3>Gcode Output Window</h3>
            <a ng-href="{{gcodeUrl}}" ng-click="downloadGcode(gcode.gcode)">download gcode</a>
        </div>
        <div class="panel-body">
            <button class="btn btn-sm" ng-click="showGcode = !showGcode" data-toggle="tooltip" data-placement="bottom" title="WARNING! This may freeze browser!">press to view gcode</button>
            <textarea ng-if="showGcode" cols="30" rows="50" ng-model="gcode.gcode"
                      class="form-control"></textarea>

        </div>
    </div>
</div>
</div>

<!--2D GCODE VISUALIZER-->
<div class="row" ng-controller="gcodeController" ng-show="gcode.gcode">
<div class="panel panel-success">
    <div class="panel-heading">
        <h3>Gcode Visualizer</h3>
    </div>
    <div class="panel-body">
        <canvas id="canvas" width="650" height="620"></canvas>
        <form>
            <div class="form-group">
                <label for="layerNumber">layerNumber</label>
                <input type="number" id="layerNumber" ng-model="layerNumber">
                <label for="progress">progress</label>
                <input type="number" id="progress" ng-model="progress">
            </div>
        </form>
    </div>
</div>
</div>

<div class="footer row">
    <p class="text-center">&copy; Michael U.B. Meding 2016</p>
</div>

</div>
</div>
</body>
</html>
\end{lstlisting}