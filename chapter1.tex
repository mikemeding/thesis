\chapter{Introduction}
3D printing, over the past few years, has become immensely popular in the home hobbyist space because of its new found availability. A hobbyist can now go and buy a do-it-yourself 3D printer kit for less than \$500. Even though the hardware to build a 3D printer is easily available, the software support leaves much to be desired. Most of the big name companies that used to hold all of the patents for 3D printers have retained the software patents but not the hardware patents. This creates a gap in knowledge between building the printer and actually running it. This proposed research is to find out what software already exists in the open source world and then try to expose this on the web in a simple and easy to use manner. This effectively will “democratize” the world of 3D printing, much in the same way that Google has democratized the way that we search the web. D'Aveni, R.’s (2015) article theorizes that this rise in the popularity of 3D printing will spur an industrial revolution as manufacturing becomes more personalized and decentralized.

To print something simple on a 3D printer, there is a multi step process that can be daunting to many first time users. The first step in the process is to either create or download a model. Creating a model can be done with any standard 3D CAD software, such as AutoCAD or SolidWorks. Downloading pre-existing models from an online repository can be done from websites such as Thingiverse or YouMagine. Once a model file is obtained, it is time to “slice” the model. Slicing is the act of taking this model file and splitting it up into many thin layers that the 3D printer can understand. This process can only be done by a dedicated slicing software which can often be complicated to use and difficult to install. Once the file has been sliced, the resulting file is a G-code file which is simply a set of movement instructions that the printer head must follow. This file is then loaded to an SD card or sent via a print server similar to the way that a normal 2D printer is networked. Once the G-code file has been loaded all that is left is to hit print either manually using the printers interface for the SD card or by hitting print on the network interface for the file that was uploaded.


%So here we are in the introduction chapter. Fun times are ahead! First, how to do citations. The basic citation is ``citet'', the textual citation, and should be done like the following sentence. Design and engineering lack such a body of work, as \citet[p. 35]{csed-guzdial} stated about the field of computing. The other kind is the parenthetical citation, ``citep,'' which looks like this. Most adults use strategies to overcome natural limitations of memory retention and recollection \citep{brown-1992}. The only difference is how the citation will render. How you use them is up to you and your desired writing style. You can even lead a sentence with a textual citation. \citet{brown-1992} conducted much of her work around the development of a learning community in lieu of the traditional didactic classroom experience. 
%
%These are functions of natbib, the details of which can be easily found by internet search.

\section{Research Focus}
\section{The RepRap Idea}
\section{Sudden Growth Of 3D Printing}
\section{Purpose of This Research}
\section{Research Objectives}
\section{Existing Technology}
% Talk about Astroprint 2 and why its not the full solution.
\section{Thesis Map}
% This is where a design for reading this research should be shown so that someone can easily figure out what they want to read about.



