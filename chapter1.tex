\chapter{Introduction}

So here we are in the introduction chapter. Fun times are ahead! First, how to do citations. The basic citation is ``citet'', the textual citation, and should be done like the following sentence. Design and engineering lack such a body of work, as \citet[p. 35]{csed-guzdial} stated about the field of computing. The other kind is the parenthetical citation, ``citep,'' which looks like this. Most adults use strategies to overcome natural limitations of memory retention and recollection \citep{brown-1992}. The only difference is how the citation will render. How you use them is up to you and your desired writing style. You can even lead a sentence with a textual citation. \citet{brown-1992} conducted much of her work around the development of a learning community in lieu of the traditional didactic classroom experience. 

These are functions of natbib, the details of which can be easily found by internet search.

\section{Research Focus}
Creating sections is easy enough. Basic LaTeX. I highly recommend the wikibook on LaTeX. It is complete and wonderful. 
 
\section{Problem Statement}
Let's do a list. Lists are great. The four questions of interest are:
\begin{itemize}
\item Do students exhibit patterns in testing and iteration? What are those patterns?
\item What characteristics of a design activity elicit specific iteration patterns?
\item What is the correlation between iteration in designing and success of the design?
\item What guidelines can be written for the creation of future activities?
\end{itemize}


\section{Approach}
This research used a laboratory study of a small sample population of middle school students. The sessions were video and audio recorded. The data was coded for many different design behaviors, informed by \citep{welch}. 

\section{Hypothesis and Contributions}
Here, each research question is framed in more detail:
\subsubsection{Do students exhibit patterns in testing and iteration?}
	It is expected that individual students will demonstrate personal trends across all activities. 
	
\subsubsection{What characteristics of a design activity elicit specific iteration patterns?}
	The design activities were designed to differ in complexity, speed of construction, and level of abstractness. These types of properties are expected to have a specific effect on iteration patterns in students, resulting in each activity having general trends that cross all students within the specific activity.
	
\subsubsection{What is the correlation between iteration in design the design process and the success of the final design?}
	Multiple sources in the literature depict iteration as critical to design success. For example, \citet{dow09} showed that, in college students, forced iteration makes an inexperienced designer just as good as non-iterating designer who has domain experience. In this study, it was expected that rate and count of iteration would strongly correlate with success. Each activity is analyzed with individual success metrics, so this hypothesis was tested within each activity separately. 

\subsubsection{What guidelines can be written for the creation of future activities?}
	Each activity was expected to result in a certain unique pattern of testing and iteration. By comparing these patterns, it would then be possible to generate recommendations on properties of the activities themselves. These guidelines could be generalized for use by educators.

\section{Rationale}
Exploring the tacit processes of novice engineers will further our understanding of human design faculties. With this study, motivations behind why students engage in certain models of behavior was explored. This study chose to focus on iteration, a single component of the engineering process. As is discussed in the Background chapter, iteration is a fundamental characteristic of the design process. This study chose to use design iteration as a lens for analysis of the student design work.
